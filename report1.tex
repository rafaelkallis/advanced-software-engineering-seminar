\documentclass{scrartcl}
\usepackage{geometry} 
\usepackage{multicol}
\geometry{top=1.4in}
 
\begin{document}
 
\title{Response Paper}
\subtitle{
    \small Software Developers’ Perceptions of Productivity (Meyer et al.) \\
    A Diary Study of Task Switching and Interruptions (Czerwinski et al.)\\
    Real-time task recognition based on knowledge workersʼ computer activities (Koldijk et al.)
}
\author{Rafael Kallis}
\maketitle

\begin{multicols}{2}
 
All papers pursue finding out the effects of task and activity switching in the knowledge worker's workflow and additionally suggest how improvements could be made.
Czerwinski et al. report complex, long duration, tasks have a significant higher "return-to" cost than other tasks.
The data has been collected from dairy studies, hence results rely on the worker's perception of the switch cost.
An explanation for the observed phenomenon, why these complex tasks are perceived more difficult to return to, is missing.
Further research should be done to gain more insights about the perception people have switching back and forth between tasks.
Meyer et al. try to shed some light on the exact question.
From their findings, observational data seems to paint a different picture than survey collected data.
The demographics of the observations seem to be quite different from the demographics of the surveys.
Most of survey participants seem to come from Switzerland and Germany, yet all observational studies were made with American and Canadian residents, a thread to validity which has not been stated.
In combination with his stated threads to validity, I can't agree nor disagree with his results.
Improvements could be made by having uniform demographics on both survey and observational based data.

Meyer et al. as well as Czerwinski et al. follow up on their findings suggesting that capturing representations of tasks, auto-categorizing them, hence visualizing one's tasks during the day, could help returning to tasks and retrospecting on productivity.
The same idea, supporting knowledge workers in their self-management by providing them overviews of performed tasks, motivates Koldijk et al. finding a suitable task classifier in her research.
The classifier can be trained using computer interaction data and tries to log a worker's tasks during the day.
As good as the trained classifiers auto-labelling success rate is, its data is solely based on activity on the computer.
It is unclear what results the classifier might produce, when also given other inputs, such as time spend in meetings, phone calls, talks with colleagues or interacting with other devices.
Having one’s vitals as an input to the classifier, or used for retrospecting productivity as identified by Meyer et al. wasn't covered in the research of Koldijk et al.

Looking over all papers again, it is still unclear to me on how these suggestions would revolutionize our workflow.
Simple todo lists, reminders, agendas and window management utilities, which are at our disposal today, do help having a better overview of progress.
Yet, the papers fail to make clear what kind of tool will supersede tools we use today.

\end{multicols}
\end{document}